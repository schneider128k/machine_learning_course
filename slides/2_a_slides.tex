\documentclass{beamer}
%
% Choose how your presentation looks.
%
% For more themes, color themes and font themes, see:
% http://deic.uab.es/~iblanes/beamer_gallery/index_by_theme.html
%
\mode<presentation>
{
  \usetheme{default}      % or try Darmstadt, Madrid, Warsaw, ...
  \usecolortheme{crane} % or try albatross, beaver, crane, ...
  \usefonttheme{structurebold}  % or try serif, structurebold, ...
  \setbeamertemplate{navigation symbols}{}
  \setbeamertemplate{caption}[numbered]
} 

\usepackage[english]{babel}
\usepackage[utf8x]{inputenc}

\title[ML]{Machine Learning}
\author{Pawel Wocjan}
\institute{University of Central Florida}
\date{Fall 2020}

\begin{document}

\begin{frame}
  \titlepage
\end{frame}

\begin{frame}{Sources for Slides}

\begin{itemize}
\item I have extensively used the machine learning materials that have been prepared by Google. 

\medskip
\footnotesize{ 
\url{https://developers.google.com/machine-learning/crash-course/}
}

\item Google has licensed these materials under the Creative Commons Attribution 3.0 License.

\medskip
\footnotesize{ 
\url{https://creativecommons.org/licenses/by/3.0/}
}
\end{itemize}
\end{frame}

% Uncomment these lines for an automatically generated outline.
%\begin{frame}{Outline}
%  \tableofcontents
%\end{frame}

\begin{frame}{Basic ML Terminology}

\begin{itemize}
\item What is (supervised) machine learning? Concisely put, it is the following:

\medskip
\emph{ML systems learn how to combine input to produce useful predictions on never-before-seen data.}

\medskip
\item Let's explore basic ML terminology machine learning terminology.
\end{itemize}
\end{frame}

%%%

\begin{frame}{Labels and features}

\begin{itemize}
\item 
A {\bf label} is what we seek to predict.

\medskip
For instance, the label could be the future price of a stock or the type of animal shown in a picture.

\medskip
\item 
A {\bf feature} $x_j$ is an input variable that is used to predict the label where $j\in\{1,\ldots,n\}$ and $n$ is the number of features.

\medskip
For instance, the feature could be the previous day's closing price of a stock or a pixel of an image.

\medskip
\item A simple machine learning project might use a single or few features, while a more sophisticated machine learning project could use millions of features. 
\end{itemize}

\end{frame}

%%%

\begin{frame}{Features for spam detection}

\begin{itemize}
\item For spam detection, for instance, the feature could include the following features:

\medskip
\begin{itemize}
\item words in the email text
\item sender's address
\item time of day the email was sent
\item email contains the phrase ``one weird trick.''
\end{itemize}
\end{itemize}

\end{frame}


%%%

\begin{frame}{Labeled and unlabeled example}

\begin{itemize}
\item An {\bf example} is a particular instance of data that is fed into a model. 

\medskip
\item We break examples into two categories:

\medskip
\begin{itemize}
\item
labeled examples

\medskip
\item
unlabeled examples
\end{itemize}
\end{itemize}

\end{frame}

%%%

\begin{frame}{Labeled example}

\begin{itemize}
\item A {\bf labeled example} includes both feature(s) and the label.

\medskip
\texttt{labeled example: \{features, label\}}: $(x, y)$

\medskip
\item We use labeled examples to train the model. 

\medskip
In our spam detector example, the labeled examples would be individual emails that users have explicitly marked as ``spam'' or ``not spam.''
\end{itemize}

\end{frame}

%%%

\begin{frame}{Unlabeled example}

\begin{itemize}
\item 
An {\bf unlabeled example} contains features but not the label:

\medskip
\texttt{unlabeled example: \{features, ?\}}: $(x, ?)$

\medskip
\item Once we've trained our model with labeled examples, we use that model to predict the label on unlabeled examples. 

\medskip
In the spam detector, unlabeled examples are new emails that humans haven't yet labeled.
\end{itemize}

\end{frame}


%%%

\begin{frame}{Training and inference}
\begin{itemize}

\medskip
\item A model defines the relationship between features and labels. 
    
\medskip
\item 
Let's highlight two phases of a model's life:

\medskip
\begin{itemize}
\item {\bf Training} means creating or learning the model. 

\medskip
You show the model labeled examples $(x, y)$ and enable it to gradually learn the relationships between the features $x$ and the label $y$ so that its predictions $\hat{y}$ are sufficiently close to the labels $y$.
        
\medskip
\item {\bf Inference} means applying the trained model to unlabeled examples. 

\medskip
You use the trained model to make useful predictions $\hat{y}$.
\end{itemize}
\end{itemize}

\end{frame}

%%%

\begin{frame}{Regression}
\begin{itemize}
\item A {\bf regression model} predicts continuous values. 

\medskip
For example, regression models make predictions that answer questions like the following:
    
\medskip    
\begin{itemize}

\medskip
\item What is the value of a house in California?

\medskip
\item What is the probability that a user will click on this ad?
\end{itemize}
\end{itemize}
\end{frame}

%%%

\begin{frame}{Classification}
\begin{itemize}
\item A {\bf classification model} predicts discrete values. For example, classification models make predictions that answer questions like the following:

\medskip
\begin{itemize}
\item Is a given email message spam or not spam?

\medskip
\item Is this an image of a dog, a cat, or a hamster?
\end{itemize}
\end{itemize}
\end{frame}

%%%

\begin{frame}{Key Terms}
\begin{itemize}
    \item classification model
    \item example
    \item feature
    \item inference
    \item label
    \item model
    \item regression model
    \item training
\end{itemize}
\end{frame}

%%%

\end{document}